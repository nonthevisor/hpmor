\input{hp-paper-type}

% N.B. All text and images should be 6mm away from all edges
\newcounter{bookpagecount}
\setcounter{bookpagecount}{\XeTeXpdfpagecount"\bookfile"}
\RequirePackage{calc} % This line must be after the previous to avoid an error
\newlength{\hppaperwidth} % no. of pages ÷ 2 × width per sheet
\setlength{\hppaperwidth}{\bookpaper * \value{bookpagecount} / \real{2.0}}
\newlength{\hpspinewidth} % paper thickness + 5mm
\setlength{\hpspinewidth}{\hppaperwidth + 5mm}
\makeatletter
\setbox\z@=\hbox{\XeTeXpdffile"\bookfile"\relax}
\newlength{\hpcoverwidth} % paper width + 8mm
\setlength{\hpcoverwidth}{\the\wd\z@ + 8mm}
\newlength{\hpcoverheight} % coverheight: paper height + 1mm
\setlength{\hpcoverheight}{\the\ht\z@ + 1mm}
\makeatother
% flapwidth: 60mm+3mm each side
\documentclass[12pt,coverwidth=\the\hpcoverwidth,coverheight=\the\hpcoverheight,spinewidth=\the\hpspinewidth,marklength=0mm,bleedwidth=5mm,flapwidth=63mm]{bookcover}

\input{hp-header}

\usepackage{contour}

\input{hp-title}

\begin{document}
{
\setbookcover{bgcolor}{whole}{color=black}
\setbookcover{fgfirst}{remark}{
\bfseries\color{blue}DUST JACKET}
\setbookcover{bgpic}{front}{cover0.jpg}
\setbookcover{fgfirst}{front}{\hptitle[\coverwidth]{\fullvolumetitle{\volumenumber}}}

% Spine
\setbookcover{fgfirst}{spine}{
\centering
\color{white}\scshape
\vspace{0.5cm}\huge \volumenumber\\[2ex]\Large
\vfill
\rotatebox[origin=c]{90}{\contour[120]{black}{\volumetitle}}
\vfill}

% Back cover
\setbookcover{fgfirst}{back}{
  \centering
  \vspace{20mm}
  \parbox{110mm}{\color{white}\Large\raggedright
  \setlength\parindent{-1em} %indent paragraphs as in dramatic dialogues
Harry: You can't DO that! \par

Minerva McGonagall: It's only a transfiguration; an animagus transformation, to be exact— \par

Harry: You turned into a cat! A SMALL cat! You violated Conservation of Energy! That's not just an arbitrary rule, it's implied by the form of the quantum Hamiltonian! Rejecting it destroys unitarity and then you get FTL signaling! And cats are COMPLICATED! A human mind can't just visualize a whole cat's anatomy and, and all the cat biochemistry, and what about the neurology? How can you go on thinking using a cat-sized brain? \par

Minerva: Magic. \par

Harry: Magic isn't enough to do that! You'd have to be a god! \par

Minerva: ... that's the first time I've been called that. \par

\raggedleft{—Harry's first encounter with magic}
}}

% Text on the front flap
\setbookcover{fgfirst}{front flap}{
\centering
\vspace{20mm}
\parbox{40mm}{\color{white}\raggedright\small
  HPMOR comes from the world of fan-fiction. With J.~K.~Rowling’s approval, it tells the story of an alternate Harry, an auto-didact rationalist who brings all his powers to bear on the strange new vistas of the world of magic.

  \bigskip The books introduce the reader to the ways of thinking Yudkowsky lays out in his blog, \url{lesswrong.com}.}}

% Text on the back flap
\setbookcover{fgfirst}{back flap}{
\centering
\vspace{20mm}
\parbox{40mm}{\color{white}\small\raggedright
  Eliezer Shlomo Yudkowsky (born September 11, 1979) is an American AI researcher and writer best known for popularising the idea of friendly artificial intelligence. He is a co-founder and research fellow at the Machine Intelligence Research Institute (MIRI), a private research nonprofit based in Berkeley, California. His work on the prospect of a runaway intelligence explosion was an influence on Nick Bostrom's \textit{Superintelligence: Paths, Dangers, Strategies.}
}}

\makebookcover
}
\end{document}
